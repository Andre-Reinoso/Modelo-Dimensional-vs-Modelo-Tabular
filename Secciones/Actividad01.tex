\section{INFORMACIÓN GENERAL} 

\begin{itemize}
\subsection{Abstract}
	\item The models facilitate the presentation of information in a standard, simple and, above all, intuitive way for users, as well as allowing access to information much faster by database administrators. Each dimensional model is composed of a table called "facts" and a set of small tables called "dimensions". Each dimension contains a primary key that "connects" to the fact table (a ratio of 1 to many). The diagrams are structures that facilitate the order of the data and, in turn, facilitate the normalization especially impossible for the optimization of the database.

\subsection{Resumen}
	\item Los modelos facilitan la presentar la información de una manera estándar, sencilla y sobre todo intuitiva para los usuarios, además de que permite accesos a la información mucho más rápida por parte de los manejadores de bases de datos. Cada Modelo Dimensional esta compuesto por una tabla llamada “de hechos” y por un conjunto de pequeñas tablas llamadas “dimensiones“. Cada dimensión contiene una llave primaria que se “conecta” a la tabla de hechos manteniendo una relación de 1 a muchos ). Los diagramas son estructuas que faciloitan el orden de los datos y a su vez facilita la normalizacion especialmente imposible para la optimizacion de las base de datos.

\subsection{Objetivos}
	\item Generales: Comprender las bases sobre el analisis de los modelos dimensionales y tabulares.

	\item Especificos: Desarrollo de cada uno de los diagramas existentes en los modelos y su objetivo especifico.

\subsection {Introduccion}

\begin{itemize}
	\item La gran revolución informacional, incrementando  la  disponibilidad  y  las  posibilidades  de  acceso  a  la  información. Han proliferado también las necesidades  de consultas más complejas para la toma de decisiones dentro de las organizaciones. La mayoría de los sistemas de gestión de bases de datos que ofrecen herramientas para realizar el proceso de data warehousing se apoyan en la tecnología orientada a filas/registros, optimizada para el procesamiento transaccional de los datos. Con el desarrollo del modelo multidimensional y diferentes alternativas de indexación no se logra eludir completamente el compromiso con el almacenamiento por filas (row-oriented).  
\\ \\
 Varios autores han defendido la contribución del almacenamiento columnar, basado  esencialmente en  la transposición  de los  ficheros para  mejorar  el desempeño  de las  consultas enfocadas hacia el análisis y la toma de decisiones.
\\ \\
Se trata de beneficiar el procesamiento analítico de los  datos,  caracterizado  por  demandas  que  requieren  el  agrupamiento  o  la  agregación  de  grandes cantidades de datos sobre unas pocas columnas, desde la perspectiva de los índices de proyección a través de las filas.
	
\end{itemize} 


\end{itemize}

